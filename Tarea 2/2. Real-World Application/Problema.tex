\documentclass{article}
\usepackage{graphicx}

\title{Problema de consumo energetico de una cinta transportadora}
\author{Juan E Osorno D.}
\date{January 2025}

\begin{document}

\section{Contexto del problema}
Supongamos que tenemos una cinta transportadora de mercancia y deseamos conocer el consumo energetico de la misma.

Sean $x,y$ y $z$ variables que describen la temperatura del ambiente en $C°$ la velocidad de la operacion en revoluciones por segundo
y la carga implicada en la operacion. Sea $w = \alpha_0  + \alpha_1 x + \alpha_2 y + \alpha_3 z$ donde $\alpha_0,\alpha_1,\alpha_2 y \alpha_3$
representan el cambio de todas estas variables. 

Un trabajador investiga la maquina y realiza una tabla de datos que muestra como cambian estas variables de la siguiente manera:
\begin{table}[h!]
    \centering
    \begin{tabular}{|c|c|c|c|c|}
    \hline
    \textbf{Punto} & $x$ (\textdegree C) & $y$ (RPM) & $z$ (kg) & $w$ (kWh) \\ \hline
    1              & 25                  & 1500      & 50       & 200       \\ \hline
    2              & 30                  & 1600      & 55       & 220       \\ \hline
    3              & 28                  & 1550      & 52       & 210       \\ \hline
    4              & 35                  & 1700      & 60       & 250       \\ \hline
    5              & 40                  & 1800      & 65       & 280       \\ \hline
    \end{tabular}
    \caption{Tabla de datos}
    \label{tab:datos}
    \end{table}

Nuestro objetivo sera encontrar las variables $\alpha_0,\alpha_1,\alpha_2 y \alpha_3$ que hagan que el consumo energetico sea minimo. Es decir
bajo que condiciones la maquina ahorra mas energia.
\section{Importancia del problema}

Supongase que usted es un empresario que desea conocer el consumo energetico de la maquina en la cual usted ha invertido.
Dado 
\end{document}