\documentclass{article}
\usepackage{amsmath}
\usepackage{graphicx}

\title{Problema de consumo energetico de una cinta transportadora}
\author{Juan E Osorno D.}
\date{January 2025}

\begin{document}

\section{Contexto del problema}
Supongamos que tenemos una cinta transportadora de mercancia y deseamos conocer el consumo energetico de la misma. 
Un trabajador investiga la maquina y realiza una tabla de datos que muestra como cambian estas variables de la siguiente manera:
\begin{table}[h!]
    \centering
    \begin{tabular}{|c|c|c|c|c|}
    \hline
    \textbf{Punto} & $x$ (\textdegree C) & $y$ (RPM) & $z$ (kg) & $w$ (kWh) \\ \hline
    1              & 25                  & 1500      & 50       & 200       \\ \hline
    2              & 30                  & 1600      & 55       & 220       \\ \hline
    3              & 28                  & 1550      & 52       & 210       \\ \hline
    4              & 35                  & 1700      & 60       & 250       \\ \hline
    5              & 40                  & 1800      & 65       & 280       \\ \hline
    \end{tabular}
    \caption{Tabla de datos}
    \label{tab:datos}
    \end{table}

Con estos datos se sospecha que el modelo que aproxima los datos es lineal.

Sean $x,y$ y $z$ variables que describen la temperatura del ambiente en $C°$ la velocidad de la operacion en revoluciones por segundo
y la carga implicada en la operacion. Sea $w = \alpha_0  + \alpha_1 x + \alpha_2 y + \alpha_3 z$ donde $\alpha_0,\alpha_1,\alpha_2 y \alpha_3$
representan el cambio de todas estas variables. 

Nuestro objetivo sera encontrar las variables $\alpha_0,\alpha_1,\alpha_2 y \alpha_3$ que predicen cual sera el consumo energetico de la maquina
dados estos datos
\section{Importancia del problema}

Supongase que usted es un empresario que desea conocer el consumo energetico de una maquina en la cual usted ha invertido. Sus 
trabajadores hacen un estudio y sacan los datos de la tabla anterior, entonces su deseo es conocer de acuerdo con esos datos cual es el
gasto energetico de la maquina y saber si se pueden cambiar las condiciones del ambiente, velocidad y peso para que dicho consumo sea mas 
minimo y asi ahorrar gastos.
\section{Problema de la relacion entre las horas de estudio y las clificaciones}

Un profesor desea estudiar la relacion entre las horas de estudio de un estudiante y la nota que obtiene. El profesor
realiza una encuesta a sus estudiantes preguntando cuantas horas estudia, y realiza una tabla con sus notas obtenidas 
en el parcial como sigue en la siguiente tabla:
\begin{table}[h!]
    \centering
    \begin{tabular}{|c|c|c|}
    \hline
    \textbf{Estudiante} & \textbf{Horas de Estudio (\(x\))} & \textbf{Calificación (\(y\))} \\
    1  & 1  & 10   \\ \hline 
    2  & 2  & 25   \\ \hline
    3  & 3  & 40   \\ \hline
    4  & 4  & 55   \\ \hline
    5  & 5  & 70   \\ \hline
    6  & 6  & 85   \\ \hline
    7  & 7  & 95   \\ \hline
    8  & 8  & 98   \\ \hline
    9  & 9  & 99   \\ \hline
    10 & 10 & 98  \\ \hline
    11  & 7  & 95   \\ \hline
    12  & 8  & 98   \\ \hline
    13  & 9  & 99   \\ \hline
    14 & 10 & 100  \\ \hline
    \end{tabular}
    \caption{Datos simulados de horas de estudio y calificación con tendencia a estabilización.}
    \label{tabla:horas_estudio_aplanada}
\end{table}

Se sospecha que el modelo que mejor aproxima estos datos es $y = \lambda(1-e^\beta x)$. Ahora para este modelo vamos a 
considerar el sistema de ecuaciones: $X\vec{b} = \vec{y}$ donde X es la matriz que cumple
\begin{center}
$ \mathbf{X} = \begin{pmatrix} 1 & e^{\beta x_1} \\ 1 & e^{\beta x_2} \\ ... & ... \\ 1 & e^{\beta x_14} \end{pmatrix} $  
\end{center}

Donde \(x_i\) son los valores de \(x\) en los datos observados en la tabla anterior. Ademas $\vec{b} = (\lambda,\beta)$ y 
$\vec{y} = (y_1,y_2,...y_14)$ son los vectores que satisfacen con el sistema de ecuaciones.

Ahora multiplicando por $X^T$ a la izquierda miembro a miembro tenemos que 

\begin{center}
    $X^TX \vec{b} = X^T \vec{y}$
\end{center}

y despejando $\vec{b}$ deberiamos obtener los $\lambda$ y $\beta$ para los cuales ese sistema se cumple y por tanto
la funcion mejor aproxima nuestra nube de puntos.

\end{document}